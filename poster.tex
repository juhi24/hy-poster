\documentclass[portrait,a1]{a0poster}
\usepackage{color,multicol}
\usepackage[english]{babel}
\usepackage[utf8]{inputenc}
\usepackage[T1]{fontenc}
\usepackage{siunitx}
\usepackage{graphicx}
\usepackage{booktabs}
\usepackage[round]{natbib}
\usepackage[font=small]{caption}
\usepackage{subcaption}
\usepackage{lipsum} % Lorem ipsum generator

\columnsep = 50pt %change this for separation of the columns

\begin{document}

\definecolor{facultyColor}{cmyk}{0,.37,.88,.02} % faculty of science
\definecolor{gray}{cmyk}{.13,.05,0,.25}

\vspace*{\fill} %some more marginal up

\begin{minipage}[t]{0.98\linewidth} % The first minipage for the logo & title
\vspace{0pt} % A trick to align the parallel minipages on top

\vspace{0.008\linewidth} % Increase the top margin

\begin{minipage}[t]{0.28\linewidth} % logo
\vspace{0pt} % Alingns the parallel minipages on top
\centering
\includegraphics[width=0.8\linewidth]{HYlogo_fac_text-en}
\includegraphics[width=0.65\linewidth]{division}

\end{minipage} % no empty line before the next begin
\begin{minipage}[t]{0.7\linewidth} % title
\vspace{0pt} % Alingns the parallel minipages on top

% More conservative title, upright and black
{\renewcommand{\baselinestretch}{0.85} % Changes the baseline skip smaller for the title
\Huge{\textbf{\textsf{The title of the poster\\ that can span\\ to multiple rows}}} % Text size for a1 posters
\par} % <- for \baselinestretch


\vspace{0.04\linewidth} % Empty space after the title

\normalsize{\textsf{\bfseries{Roger R.\ Researcher and Paula P.\ Professor}}} % Text size for a1 posters

\textcolor{gray}{\textsf{\bfseries{Department of Physics, Helsinki University}}}
%
\end{minipage}
\end{minipage}


\vfill %some more marginal between header and text

\begin{multicols}{3} %change this for different number of columns

\section{Introduction}

\lipsum[1]

\section{Section}

\lipsum[2-3]

\section{Another section}

\lipsum[4-7]

\section{Conclusions}

\begin{itemize}
\item \lipsum[8]
\item \lipsum[9]
\end{itemize}

\end{multicols}

\vfill %some more marginal in the end

\end{document}